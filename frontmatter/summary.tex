%!TEX root = ../thesis.tex

%This is the Summary
%%=========================================
\cleardoublepage
\addcontentsline{toc}{section}{Abstract}
\section*{Abstract}
An increased modal share of bicycle traffic is a key mechanism to reduce emissions and solve traffic-related problems.
However, a lack of comfort and (perceived) safety keeps people from using their bicycles more frequently.
To improve comfort and safety in bicycle traffic, city planners need (a), an overview of accidents, near miss incidents, and bike routes, (b), an overview of the cycling infrastructure's maintenance state, and (c), a way to study the effect of potential changes in the cycling infrastructure before implementing them in the real life.
However, these needs, which are also of interest to cyclists and policy makers, are currently not being met satisfactorily.

In this thesis, we present three main contributions to solve these problems.
First, we show SimRa, a crowdsourcing-based citizen science project to record cycling trips and near miss incidents.
As part of this contribution, we also present CycleSense, an approach based on Deep Learning to automatically detect near miss incidents from recorded cycling trip data.
Knowing the hazardous bicycle traffic hotspots, policy makers and city planners can approve and implement safety improvements for cyclists.
Cyclists, on the other hand, can use this information to circumvent these spots.
Second, we describe an approach for deriving the road surface quality from cycling trip data.
Our approach uses data produced by the inertial measurement unit of the smartphone to calculate the smoothness of the road surface from the vibrations caused by traversing on the road with a bicycle. 
We then show how our approach can easily be integrated into a cycling trip recording app with SimRa as a case study.
This contribution also contains a visualization of the cycling infrastructure's smoothness and helps policy makers, city planners and cyclists in the same way as our first contribution.

Third, we improve the cyclist simulation model of the urban traffic simulation software SUMO.
For this, we first show that the default cyclist model of SUMO is not realistic by comparing it to cyclist behavior in the SimRa dataset.
We then split the cycling trips in the SimRa dataset into three parts for slow, medium, and fast cyclists and implement our findings as a plugin that can be added to SUMO for a more realistic cyclist simulation.
This new cyclist simulation can provide a strong tool for city planners to better to study effects of potential changes in the traffic infrastructure. 

\afterpage{\null\thispagestyle{empty}\newpage}
\addcontentsline{toc}{section}{Kurzdarstellung}
\section*{Kurzdarstellung}
Eine Erhöhung des Fahrradverkehrs trägt zur Verringerung der Treibhausgasemissionen und zur Lösung verkehrsbedingter Probleme bei.
Mangelnder Komfort und (gefühlte) Unsicherheit halten die Menschen jedoch davon ab, öfter Fahrrad zu fahren.
Um den Komfort und die Sicherheit im Radverkehr zu erhöhen, benötigen Stadtplanende (a) einen Überblick über Unfälle und Gefahrensituationen, (b) einen Überblick über den Wartungszustand der Radverkehrsinfrastruktur und (c) eine Möglichkeit, die geplanten Änderungen an der Radverkehrsinfrastruktur zu testen, bevor sie umgesetzt werden.
In dieser Arbeit präsentieren wir drei Beiträge zur Lösung dieser Probleme.

Zunächst stellen wir SimRa vor, ein Crowdsourcing-basiertes Citizen-Science-Projekt zur Erfassung von Gefahrensituationen im Radverkehr.
Im Rahmen dieses Beitrags stellen wir auch CycleSense vor, einen auf Deep-Learning basierenden Ansatz zur automatischen Erkennung von Gefahrensituationen aus aufgezeichneten Radfahrdaten.
Stadtplanende und politische Entscheidungstragende können die mit SimRa aufgedeckten gefährlichen Stellen im Radverkehr mit Baumaßnahmen verbessern.
Radfahrende können diese Informationen nutzen, um diese Stellen zu umfahren.

Im zweiten Beitrag beschreiben wir, wie wir die Straßenoberflächenqualität aus Fahrradfahrtdaten ermitteln.
Dazu verwenden wir Daten, die von den Bewegungssensoren des Smartphones erzeugt werden, um Unebenheiten der Straßenoberfläche anhand der Fahrtvibrationen zu berechnen.
Wir zeigen am Beispiel der SimRa-App, wie unser Ansatz in eine App zur Aufzeichnung von Fahrradrouten integriert werden kann.
Dieser Beitrag enthält auch eine Visualisierung der Ergebnisse und hilft sowohl Stadtplanenden als auch Radfahrenden.

Im dritten Beitrag verbessern wir das Radfahrersimulationsmodell der Verkehrssimulationssoftware SUMO.
Wir zeigen die Schwächen des Radfahrermodells von SUMO, indem wir es mit dem Verhalten von Radfahrenden im SimRa-Datensatz vergleichen.
Anschließend teilen wir die Radfahrten im SimRa-Datensatz in drei Teile für langsame, mittlere und schnelle Radfahrende auf und implementieren einen realistischeres Modell für SUMO.
Dies kann ein starkes Werkzeug für Stadtplanende sein, um ihre geplanten Änderungen an der Verkehrsinfrastruktur besser zu testen.
\afterpage{\null\thispagestyle{empty}\newpage}
