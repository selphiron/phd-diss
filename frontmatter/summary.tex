%!TEX root = ../thesis.tex

%This is the Summary
%%=========================================
\cleardoublepage
\addcontentsline{toc}{section}{Abstract}
\section*{Abstract}

An increased modal share of bicycle traffic is a key mechanism to reduce emissions and solve traffic-related problems.
However, a lack of (perceived) safety keeps people from using their bikes more frequently.
To improve safety in bicycle traffic, city planners need an overview of accidents, near miss incidents, and bike routes.
Such information, however, is currently not available.
In this paper, we describe SimRa, a platform for collecting data on bicycle routes and near miss incidents using smartphone-based crowdsourcing.
We also describe how we identify dangerous near miss hotspots based on the collected data and propose a scoring model.



\newpage
\addcontentsline{toc}{section}{Zusammenfassung}
\section*{Kurzdarstellung}

Eine Erhöhung des Fahrradverkehrs ist ein wichtiges Instrument zur Verringerung der Emissionen und zur Lösung verkehrsbedingter Probleme. Allerdings hält ein Mangel an (gefühlter) Sicherheit die Menschen davon ab, ihr Fahrrad häufiger zu benutzen. Um die Sicherheit im Radverkehr zu verbessern, benötigen Stadtplaner einen Überblick über Unfälle, Gefahrensituationen und Radwege. Solche Informationen sind jedoch derzeit nicht verfügbar. In diesem Beitrag beschreiben wir SimRa, eine Plattform zur Sammlung von Daten über Fahrradrouten und Gefahrensituationen mithilfe von Smartphone-basiertem Crowdsourcing. Außerdem beschreiben wir, wie wir auf der Grundlage der gesammelten Daten gefährliche Hotspots für Gefahrensituationen identifizieren und ein Scoring-Modell vorschlagen.
