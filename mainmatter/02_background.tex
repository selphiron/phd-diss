\cleardoublepage
\chapter{Background (rename?)}
\label{cha:background}

\section{Crowdsensing}
\label{sec:crowdsensing_background}

\section{Citizen Science}
\label{sec:citizen_science_background}

\section{Deep Learning}
\label{sec:deep_learning_background}

\section{SUMO}
\label{sec:sumo_background}
% sumo in general
SUMO is an open source traffic simulation tool that offers macroscopic as well as microscopic simulation of vehicle mobility~\cite{lopez2018microscopic}.
SUMO includes models for different types of ``vehicles'', including, among others, cars, bicycles, and even pedestrians.
Due to its large feature set, it has become the de-facto standard for traffic simulation and is used even beyond the transport community, e.g., \cite{beilharz2021towards}.

% network
Traffic scenarios are, among other things, defined by road networks and vehicle traffic.
The road network includes roads and their (sub-)lanes as well as exclusive lanes for cyclists and pedestrians, or road-side infrastructure such as traffic lights.
Furthermore, connections between these lanes and traffic lights can be configured.

% vtypes
When modeling vehicle traffic, users specify demand for a specific road segment per vehicle type and can adjust vehicle-specific parameters of SUMO's simulation model to control their respective behavior.
In general, vehicle parameters are usually specified in the vehicle type declaration (\textit{vType}), applying the changes to all instances of the respective \textit{vType}, e.g., to all cars.
An alternative, however, is to obtain multiple \textit{vType} realizations which typically differ in at least one parameter by using so-called \textit{vTypeDistributions}.
This way, when spawning a new vehicle, SUMO randomly picks a specific \textit{vType} from the \textit{vTypeDistribution} and instantiates the vehicle's parameters accordingly, e.g., cars can thus have individual maximum velocities.

% vehicle behavior
In SUMO, vehicle behavior is, among other things, defined by
\textit{Car Following (CF) models} for the longitudinal kinematic behavior,
\textit{Lane Change (LC) models} for the lateral kinematic behaviour,
and \textit{junction models} for the behavior at junctions and intersections.

% bicycles
Despite including several of these models for cars and trucks, SUMO does not provide a dedicated movement model for cyclists.
Instead, cyclists are simulated by modeling them either as slow cars or fast pedestrians.
Both of these approaches use movement models of the corresponding vehicle type and adapt their respective shape and kinematic characteristics (e.g., velocity and acceleration profiles) to match cyclists.
While this is obviously a rough approximation, it is unlikely to reflect the behavior of real-world cyclists~\cite{grigoropoulos2019modelling}.

\section{Surface Quality}
\label{sec:surface_quality_background}
Traffic departments frequently analyze the road surface quality to find out road segments that need to be repaired to increase the traffic safety.
There are mainly three types of methods used for, which we briefly introduce here.
\textit{Profilographs} are one of the oldest devices used to measure road surface quality.
They consist of a profiling wheel in the center and multiple support wheels held together by a rigid frame.
They measure the road quality by tracking the vertical movement of the profiling wheel.
The higher the vertical movement of that wheel, the worse the surface quality of the road.
This method of measuring surface quality is very slow, since the profilograph has to be pulled very slowly by another vehicle.
\textit{Scanner-based systems} are more sophisticated and rely either on accelerometers measuring the vibrations caused by driving on a specific surface, lasers scanning the surface in front or beneath the vehicle, cameras making photos later to be analyzed with the help of computer vision and photogrammetry, or a combination of the aforementioned methods.
This system provides the highest resolution and accuracy, however it is also the most costly and complex one.
More recently, \textit{smartphone-based systems} are gaining traction due to their cost-efficiency.
A smartphone is being attached inside the car and the vibrations caused by the road surface are recorded.
This is the least accurate system, since the measured vibrations are very indirect, due to car tires and suspension.

However, these systems are impractical for measuring bicycle road surface quality, because cars are not suited to drive on bicycle roads and these systems are too costly.
Hence, we propose a novel system using crowdsourced smartphone bicycle ride data.
