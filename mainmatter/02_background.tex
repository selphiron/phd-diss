\cleardoublepage
\chapter{Background (rename?)}
\label{cha:background}

\section{Crowdsensing and Citizen Science}
\label{sec:crowdsensing_background}
Other edge-assisted data analysis work includes Mei et al.~\cite{mei2017ultraviolet}, who measure UV radiation based on smartphone cameras and crowdsensing,
Cao et al.~\cite{cao2015fast} who use motion sensors to detect strokes in patients falling to the ground, and
Pham et al.~\cite{pham2015a} implement a smart parking system by equipping parking spots with RFID chips to track their occupancy state.
There are also multiple publications on placing different components of an edge-to-cloud data processing pipeline in various use cases, e.g.,~\cite{paper_lujic_intrasafed5g,paper_pfandzelter_zero2fog,hattab2019optimized}.

\section{Deep Learning}
\label{sec:deep_learning_background}
One of the most common tasks for \ac{tsc} are Natural Language Processing, speech recognition, and audio recognition in general.
Another field that deals with this problem is \acf{har} which is concerned with identifying the specific activity of a human based on sensory time series data.
Time windows of a few seconds are classified into activity categories (e.g., walking, sitting, running, lying).
While various feature extraction and pattern recognition methods have been successfully applied in the past in this context~\cite{bulling2014tutorial}, those approaches have constraints such as hand-crafted feature extraction, being able to only learn shallow features~\cite{yang2015deep}, or the requirement for large amounts of well-labeled data for model training~\cite{wang2019deep}.
\acl{dl} techniques and more specifically \acp{cnn} have recently proven to overcome these issues and deliver convincing results in the context of \ac{har}~\cite{wang2019deep,Ronao2015}.
Also, implementations of \acp{rnn} such as \ac{lstm}~\cite{Tao2016,yao2017deepsense} have proven to be successful.
Yao et al.\ \cite{yao2017deepsense} use a combination of \acp{rnn} and \acp{cnn} on top of a \ac{sf} approach after preprocessing their input data using a \ac{dft}.

\section{SUMO}
\label{sec:sumo_background}
% sumo in general
SUMO is an open source traffic simulation tool that offers macroscopic as well as microscopic simulation of vehicle mobility~\cite{lopez2018microscopic}.
SUMO includes models for different types of ``vehicles'', including, among others, cars, bicycles, and even pedestrians.
Due to its large feature set, it has become the de-facto standard for traffic simulation and is used even beyond the transport community, e.g., \cite{beilharz2021towards}.

% network
Traffic scenarios are, among other things, defined by road networks and vehicle traffic.
The road network includes roads and their (sub-)lanes as well as exclusive lanes for cyclists and pedestrians, or road-side infrastructure such as traffic lights.
Furthermore, connections between these lanes and traffic lights can be configured.

% vtypes
When modeling vehicle traffic, users specify demand for a specific road segment per vehicle type and can adjust vehicle-specific parameters of SUMO's simulation model to control their respective behavior.
In general, vehicle parameters are usually specified in the vehicle type declaration (\textit{vType}), applying the changes to all instances of the respective \textit{vType}, e.g., to all cars.
An alternative, however, is to obtain multiple \textit{vType} realizations which typically differ in at least one parameter by using so-called \textit{vTypeDistributions}.
This way, when spawning a new vehicle, SUMO randomly picks a specific \textit{vType} from the \textit{vTypeDistribution} and instantiates the vehicle's parameters accordingly, e.g., cars can thus have individual maximum velocities.

% vehicle behavior
In SUMO, vehicle behavior is, among other things, defined by
\textit{Car Following (CF) models} for the longitudinal kinematic behavior,
\textit{Lane Change (LC) models} for the lateral kinematic behaviour,
and \textit{junction models} for the behavior at junctions and intersections.

% bicycles
Despite including several of these models for cars and trucks, SUMO does not provide a dedicated movement model for cyclists.
Instead, cyclists are simulated by modeling them either as slow cars or fast pedestrians.
Both of these approaches use movement models of the corresponding vehicle type and adapt their respective shape and kinematic characteristics (e.g., velocity and acceleration profiles) to match cyclists.
While this is obviously a rough approximation, it is unlikely to reflect the behavior of real-world cyclists~\cite{grigoropoulos2019modelling}.

\section{Surface Quality}
\label{sec:surface_quality_background}
Traffic departments frequently analyze the road surface quality to find out road segments that need to be repaired to increase the traffic safety.
There are mainly three types of methods used for, which we briefly introduce here.
\textit{Profilographs} are one of the oldest devices used to measure road surface quality.
They consist of a profiling wheel in the center and multiple support wheels held together by a rigid frame.
They measure the road quality by tracking the vertical movement of the profiling wheel.
The higher the vertical movement of that wheel, the worse the surface quality of the road.
This method of measuring surface quality is very slow, since the profilograph has to be pulled very slowly by another vehicle.
\textit{Scanner-based systems} are more sophisticated and rely either on accelerometers measuring the vibrations caused by driving on a specific surface, lasers scanning the surface in front or beneath the vehicle, cameras making photos later to be analyzed with the help of computer vision and photogrammetry, or a combination of the aforementioned methods.
This system provides the highest resolution and accuracy, however it is also the most costly and complex one.
More recently, \textit{smartphone-based systems} are gaining traction due to their cost-efficiency.
A smartphone is being attached inside the car and the vibrations caused by the road surface are recorded.
This is the least accurate system, since the measured vibrations are very indirect, due to car tires and suspension.

However, these systems are impractical for measuring bicycle road surface quality, because cars are not suited to drive on bicycle roads and these systems are too costly.
Hence, we propose a novel system using crowdsourced smartphone bicycle ride data.

\section{Bicycle Traffic Safety}
\label{sec:bicycle_traffic_safety_background}
Studies on safety in bicycle traffic often rely on crowdsourcing, e.g., using Strava\footnote{https://www.strava.com/} as a data source.
While the Strava data are heavily biased towards recreational trips, some studies rely on them for analyzing various aspects of safety in bicycle traffic, e.g.,~\cite{Hochmair2019,Ferster2021strava}.
Blanc and Figliozzi~\cite{blanc2016modeling, blanc2017safety} study the perceived comfort levels of cyclists based on routes taken.
Wu et al.~\cite{wu2018predicting} predict perceived bicycle safety by combining data from, e.g., OpenStreetMap\footnote{https://www.openstreetmap.com/}, crime statistics, and parking volumes.
Similarly, Yasmin and Eluru~\cite{Yasmin2016} use various open data sources to characterize and provide estimates for bicycle safety in urban areas.
He et al.~\cite{He2018} analyze trajectories of a bike sharing service to detect events of illegally parked vehicles which frequently affect cycling safety.
Figliozzi et al.~\cite{figliozzi2019evaluation} evaluate video recordings to identify safety and delay related problems in bicycle and bus traffic.
Unsurprisingly, bicycles crossing bus lanes can cause slight delays for the buses.

In addition, Kobana et al.~\cite{kobana2014detection} focus on the detection of road damage via smartphone data, and Candefjord et al.~\cite{candefjord2014using} evaluate the development of a crash detection algorithm for cycling accidents.

Most closely related to our work, Aldred and Goodman~\cite{aldred2018predictors} analyze near-miss incidents using road diaries of cyclists and our previous work~\cite{karakaya2020simra} proposed the SimRa platform as well as the heuristic for detecting (near-miss) incidents.
A first extension of our own detection approach~\cite{sanchez2020detecting} developed an \ac{ann} to solve the problem of incident detection on the SimRa data set.
While offering some improvement over the original heuristic, CycleSense clearly outperforms that approach.
Furthermore, Ibrahim et al.~\cite{ibrahim2021cycling} discuss the potential of detecting incidents using image and video data in combination with computer vision techniques.
