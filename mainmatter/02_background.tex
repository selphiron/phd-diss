\cleardoublepage
\chapter{Background and Related Work}
\label{cha:background}
The main contributions of this thesis revolve around the SimRa project (see section~\ref{subsec:simra}), which is a crowdsourcing/crowdsensing citizen science application, with the goal of improving cycling comfort and cycling safety.
In this chapter, we go through fundamental topics that are needed to better understand the nature of the SimRa project and thus, the contributions outlined in chapter~\ref{cha:contributions}.
For this, we first outline which factors mainly influence cycling comfort and give an overview of recent studies from that research area in section~\ref{sec:cycling_comfort_background}.
We then continue with cycling safety in section~\ref{sec:cycling_safety_background}, which is related to cycling comfort.
There, we also differentiate between bicycle safety and bicycle traffic safety.
In chapter~\ref{sec:crowdsensing_crowdsourcing_background} denote similarities and differences between crowdsensing and crowdsourcing, so that we can later, in section~\ref{subsec:simra}, better categorize the SimRa project somewhere between them.
Lastly, to complete this chapter, we take a brief look into the field of citizen science, namely what makes a project a citizen science project, which benefits and challenges it brings and how the related work makes use of it in chapter~\ref{sec:citizen_science_background}. 

\section{Cycling Comfort}
\label{sec:cycling_comfort_background}
Other edge-assisted data analysis work includes Mei et al.~\cite{mei2017ultraviolet}, who measure UV radiation based on smartphone cameras and crowdsensing,
Cao et al.~\cite{cao2015fast} who use motion sensors to detect strokes in patients falling to the ground, and
Pham et al.~\cite{pham2015a} implement a smart parking system by equipping parking spots with RFID chips to track their occupancy state.
There are also multiple publications on placing different components of an edge-to-cloud data processing pipeline in various use cases, e.g.,~\cite{lujic2021increasing,pfandzelter2021zero,hattab2019optimized}.


\section{Cycling Safety}
\label{sec:cycling_safety_background}
Other edge-assisted data analysis work includes Mei et al.~\cite{mei2017ultraviolet}, who measure UV radiation based on smartphone cameras and crowdsensing,
Cao et al.~\cite{cao2015fast} who use motion sensors to detect strokes in patients falling to the ground, and
Pham et al.~\cite{pham2015a} implement a smart parking system by equipping parking spots with RFID chips to track their occupancy state.
There are also multiple publications on placing different components of an edge-to-cloud data processing pipeline in various use cases, e.g.,~\cite{lujic2021increasing,pfandzelter2021zero,hattab2019optimized}.


\section{Crowdsensing and Crowdsourcing}
\label{sec:crowdsensing_crowdsourcing_background}
Crowdsensing and crowdsourcing have emerged as successful approaches to gather big amounts of data in a short time and solve tasks in a fast and cost-efficient way respectively.


\section{Citizen Science}
\label{sec:citizen_science_background}