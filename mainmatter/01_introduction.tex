\cleardoublepage
\chapter{Introduction}
\label{cha:introduction}
The global climate crisis is and will be one of mankind's most vital challenges in the first half of the 21\textsuperscript{st} century and the effects thereof are already observable.
The frequency of extreme weather conditions and natural disasters such as long periods of precipitation leading to floods, or a lack of precipitation leading to droughts as well as heavy winds leading to storms is rising.
Such disasters have short and long term effects on the agricultural production of the affected areas leading to famine, conflicts and mass migration.
To prevent further aggravation of the effects, and stop or at least slow down the global warming, it is important to understand the root cause, so possible solutions can be derived and implemented. 
According to the \ac{ipcc}, the surface temperature of the globe has risen by 1.1°C during the period 2011-2020 when compared to the period of 1850-1900~\cite{lee2023climate}.
There is a consensus that this increase of the surface temperature is caused by anthropogenic emissions stemming from human activities such as deforestation and the burning of fossil fuels and thus releasing $CO_{2}$ and $NO_{x}$ to the atmosphere~\cite{archer2010climate}.
If the current trend of green house gas emissions continues, the Paris Agreement's~\cite{un2015paris} goal to keep the increase of the surface temperature of the globe under 2°C will not be met~\cite{noah2023data}.
That means that to stop or at least to slow down the global warming, a decrease of greenhouse gas emissions is needed as much and as fast as possible.
Reports published by the European Environment Agency of the European Union and the \ac{ipcc} of the United Nations indicate that transportation is one of the categories with substantial CO2 emissions~\cite{lee2023climate2}.
Cars cause over 70\% of the green house gas emissions, indicating that there is a big potential for savings.
More and more societies worldwide are aware of this problem and search for solutions to adapt to the new changes that comes with the climate change and try to decrease the impact of the global climate crisis.
This is why cities worldwide try to increase the modal share of bicycle traffic and by doing so, decrease the usage of motorized private transport.
The heavy usage of motorized private transport in urban areas also cause other problems.

First, there are the motor vehicle traffic crashes with an annual death toll of over 1.3 million and an annual injury toll of over 78.2 million world wide~\cite{bhalla2014transport}.
Additionally, these crashes can cause long-lasting psychological traumas for everyone involved, such as witnesses or police officers, health workers and firefighters that have to document the crash, help injured people and rescue people who are stuck in the car wreck.

Second, there are health issues that are indirectly caused by motorized private transport.
Transportation is one of the main causes for air pollution in urban areas in Europe~\cite{european2019european} and it is estimated, that around 8 million people die due to health issues that are connected to air pollution every year~\cite{forouzanfar2016global}. 
Motorized private transport causes air pollution in various ways.
Cars with an internal combustion engine produce (green house) gases such as CO2 or NOx as a byproduct from burning fossil fuels and particulate matter from the friction of the brakes and other car parts, as well as the friction between the tires and the street.

Third, noise pollution, to which car traffic undoubtedly also contributes significantly, also causes serious health issues~\cite{khreis2016health}.
Sleep and stress disorders, adverse reproductive outcomes are just some of the problems related to traffic noise.

Other problems of motorized private transport in urban areas and car-centric city planning are urban sprawl, which leads to less walkable cities~\cite{patacchini2009urban} and thus a more sedentary lifestyle, which in turn causes health issues stemming from a lack of physical activities, soil sealing, which damages biodiversity and the ecosystem~\cite{tobias2018soil} and many more ~\cite{bozovic2021non,pritchard2022maas,mayers2020whose}.  


\section{Problem Statement}
\label{sec:problem}
Since cities were developed with the motorized private transport as the main transportation mode in mind, the traffic infrastructure highly favors the usage of cars.
Because of that, cyclists often have a hard time commuting, since they have to cycle on roads that were made for cars, which often puts them into dangerous and stressful situations.
Although crash statistics do not convey an increased danger for cyclists~\cite{juhra2012bicycle}, studies have shown, that the perceived safety of commuting with a bicycle is very low and one of the main reasons why people prefer other transportation modes to cycling~\cite{horton2016fear}.
This poses a challenge to people who want to cycle, city planners, and politicians alike.
People who want to cycle have to overcome their (perceived) lack of safety and comfort while commuting by bicycle to not feel forced to travel by car. 
It is up to city planners to improve the traffic infrastructure for cyclists, but it is not easy to pinpoint the most problematic areas, that need to be prioritized for them to be fixed, which is needed because of the limited resources available to the city planning department.
Besides, city planners are bound by city planning codes and rules, which oftentimes also reflect the car-centric approach and hinders them to change the traffic infrastructure in such a way that would benefit cyclists.
This passes the responsibility to the politicians, who need to change the city planning codes and rules and also increase the funds for bicycle infrastructure.
For this, however, they need more than anecdotal evidence, that the perceived safety in bicycle traffic is low, so that they can change the legislation and increase funding in favor of bicycle infrastructure.

To enable city planners and politicians to do evidence-based decision-making, they need a) to be aware of the problems in bicycle traffic, b) that a lot of people are interested in an improved bicycle traffic safety and c) to know where and how to improve the bicycle infrastructure.
Cyclists need good bicycle infrastructure that increase both safety and comfort of cycling, so that more and more people choose the bicycle over other modes of transport.
If the existing bicycle infrastructure is not good enough, they need to know which street sections are dangerous so that they can avoid them or be extra vigilant, when they have to pass through them.
Additionally, it would motivate cyclists to know that the problem is worked on and to keep the demand for a good bicycle infrastructure high. 

\section{Contributions of this Thesis}
\label{sec:contributions}
In this thesis we addresses the problems mentioned above by fulfilling three needs:
First, we conceptualize and implement a bicycle ride data gathering platform, that also collects information about near miss incidents.
Second, we derive the surface quality of bicycle roads.
Third, we improve the bicycle simulation of a widely used traffic simulation software.

\subsection{SimRa: A Platform for Gathering Bicycle Ride Data and Detecting Near Miss Incidents in Bicycle Traffic}
\label{subsec:simra_contribution}
The SimRa\footnote{SimRa is a German acronym for ``Sicherheit im Radverkehr´´, which translates to Safety in Bicycle Traffic.} platform comprises all things related to the collection, storage, and analysis of crowdsensed cycling data.
The data acquisition relies on a free app, which is available on both major smartphone operating systems Android and iOS, installed on the smartphones of participating cyclists.
This app collects data and detects near miss incidents during rides, lets users add comments or labels, and anonymizes the data before uploading it to the SimRa servers.
Initially, the detection of near miss incidents used a heuristic, where the accelerometer data of the recorded ride were analyzed and the biggest differences in consecutive values were considered a near miss incident.
We then built a second near miss incident detection option that relies on deep learning model.
The anonymized data comprises information on cyclist routes, near miss incidents, user demographics, as well as some aggregated ride statistics.
Finally the collected data gets processed and analyzed continuously to gain insights into dangerous street segments and intersections.
For this, we have developed one approach for interactive exploratory data analysis based on a web application and one for confirmatory data analysis which automatically derives a ‘‘dangerousness’’ score per street segment and intersection. 

We have published this contribution in~\cite{karakaya2020simra,karakaya2022cyclesense}.

\subsection{Deriving Road Surface Quality from Bicycle Ride Data}
\label{subsec:road_surface_contribution}
Deriving road surface quality is one way how we leverage the bicycle ride data gathered with SimRa.
The inclusion of the accelerometer sensor readings into the ride data recordings was mainly to be able to detect near miss incidents based on that data, since the main idea behind that is, that near miss incidents trigger sudden movements of the cyclists and thus can be recognized in the motion sensor readings.
However, a byproduct of this approach is, that the SimRa dataset also contains information about the surface quality of the roads the ride was recorded on.
While the automatic detection of near miss incidents views the vibrations stemming from the uneven surface of the roads as noise data, here we can use them to derive the ``bumpiness´´ of the road.
One major challenge, though, is the heterogeneity  of the user base who contribute to the SimRa dataset.
Different bicycle types and the place of the smartphone during the ride have both a big impact on how intensive the vibrations caused by the road surface are detected by the accelerometer sensors.
The ride data contain information about the bicycle type and the position of the smartphone during the ride, however, these information can be given voluntarily by the user.
Heterogeneous cyclists means also different styles of cycling.
While some may cycle rather quiet and relaxed, others may tend to cycle fast and sporty, which can also influence the resulting motion sensor readings.
Additionally, a big variety of different smartphone models result in many different hard- and software, when it comes to reading motion sensor data.
The aforementioned aspects can result in different ``bumpiness´´ values, from many rides that go through the same road.
This is why our approach considers the relative road surface quality that makes it possible to successfully compare different streets with each other.

We have published this contribution in~\cite{karakaya2023crowdsensing}.

\subsection{Cyclist Model for SUMO}
\label{subsec:sumo_contribution}
The SimRa dataset contains the GPS trace of the ride as well as the timestamp for each tracked GPS point.
The main reasoning to include that data is to be able to show the recorded ride to the user and to place near miss incidents at the correct position.
Additionally, by knowing where the ride has been through, we can identify dangerous and safe areas in bicycle traffic.
However, we can also extract the longitudinal movement profile, that is the acceleration, deceleration and maximum velocity of cyclists, as well as their behavior at intersections.
With this, we improve the bicycle simulation model of SUMO~\cite{lopez2018microscopic}, which simulates bicycles as slower cars or faster pedestrians by default, which is obviously far from being realistic.
To do so, we first separate the rides into slow, medium and fast rides to distinguish between slow, medium and fast cyclists.
A comparison of the longitudinal movement and the left-turn behavior at intersections shows how unrealistic the default bicycle simulation of SUMO is.
Based on the three groups of rides, we each create a model for slow, medium and fast cyclists in SUMO and show the improvement compared to the default bicycle model of SUMO.

We have published this contribution in~\cite{karakaya2022realistic,karakaya2023achieving}.

\section{Structure of this Thesis}
\label{sec:structure}

This thesis consists of three parts. Part I: Foundations begins with this chapter and proceeds with background information on the topics of  and an overview of the related work that are needed for a better understanding of the rest of this thesis. 