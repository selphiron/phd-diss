\cleardoublepage
\chapter{Introduction}
\label{cha:introduction}
The global climate crisis is one of the main challenges mankind has to overcome.
More and more societies worldwide are aware of this problem and search for solutions to decrease the impact of the global climate crisis.
The main attack point we have is to reduce greenhouse gas emissions such as $CO_{2}$ and $NO_{x}$.
This is why cities worldwide try to increase the modal share of bicycle traffic and by doing so, decrease the usage of individual motorized transport, which is one of the main greenhouse gas emitters in urban areas (CITATION NEEDED).


\section{Problem Statement}
\label{sec:problem}

\section{Contributions}
\label{sec:contributions}

\section{Thesis Structure}
\label{sec:structure}

Function-as-a-Service (FaaS) is a cutting-edge service model that has developed with the current advancement of cloud computing.
Cloud functions allow custom code to be executed in response to an event.
In most cases, developers need only worry about their actual code, as event queuing, underlying infrastructure, dynamic scaling, and dispatching are all handled by the cloud provider~\cite{Baldini2017-zf,McGrath2017-or}.

This scalable and flexible event-based programming model is a great fit for IoT event and data processing.
Consider as an example a connected button and lightbulb.
When the button is pressed it sends an event to a function in the cloud which in turn sends a command to the lamp to turn on the light.
The three components are easily connected and only the actual function code would need to be provided.
Thanks to managed FaaS, this approach also scales from two devices to thousands of devices without any additional configuration.

Current FaaS platforms do provide these benefits for the IoT, however, using them in this way is inefficient.
Sending all events and data to the cloud for processing leads to a high load on the network and high response latency~\cite{paper_zhang_cloud_is_not_enough_GDP,paper_bermbach_fog_computing}.
It is much more efficient to process IoT data closer to their service consumers such as our lightbulb and button, as is the idea of fog computing~\cite{Bermbach2020-sf,Pfandzelter2019-so}.
Positioned in the same network, our button may send its event to an edge function placed, for example, on a common gateway.
This also introduces additional transparency about data movement within the network and alleviates some security concerns about cloud computing~\cite{Bonomi2012-if,paper_bermbach_fog_computing}.

Currently, however, there are no open FaaS platforms that are built specifically for IoT data processing at the edge.
State-of-the-art platforms are instead built for powerful cloud hardware, for web-based services, or are proprietary software that is not extensible.

We therefore make the following contributions in this thesis:

\begin{enumerate}
  \item We discuss the unique challenges of IoT data processing and edge computing and derive requirements for an edge FaaS platform (Chapter~\ref{cha:background})
  \item We introduce \textit{tinyFaaS}, a novel FaaS platform architecture that addresses the requirements we have identified (Chapter~\ref{cha:systemdesign})
  \item We evaluate \textit{tinyFaaS} through a proof-of-concept prototype and a number of experiments in which we compare it to state-of-the-art FaaS platforms, namely Kubeless and Lean OpenWhisk (Chapter~\ref{cha:evaluation})
\end{enumerate}
