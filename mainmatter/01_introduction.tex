\cleardoublepage
\chapter{Introduction}
\label{cha:introduction}
The global climate crisis is and will be one of mankind's most vital challenges in the first half of the 21st century and the effects of thereof are already observable.
The frequency of extreme weather conditions and natural disasters such as long periods of precipitation leading to floods, or a lack of precipitation leading to droughts as well as heavy winds leading to storms is rising.
Such disasters have short and long term effects on the agricultural production of the affected areas leading to famine, conflicts and mass migration.
To prevent further aggravation of the effects, and stop or at least slow down the global warming, it is important to understand the root cause, so possible solutions can be derived and worked on. 
According to the \ac{ipcc}, the surface temperature of the globe has risen by 1.1°C during the period 2011-2020 when compared to the period of 1850-1900~\cite{lee2023climate}.
There is a consensus that this increase of the surface temperature is caused by anthropogenic emissions stemming from human activities such as deforestation and the burning of fossil fuels and thus releasing $CO_{2}$ and $NO_{x}$ to the atmosphere~\cite{archer2010climate}.
If the current trend of green house gas emissions continues, the Paris Agreement's~\cite{un2015paris} goal to keep the increase of the surface temperature of the globe under 2°C will not be met~\cite{noah2023data}.
That means that to stop or at least to slow down the global warming, a decrease of greenhouse gas emissions is needed as much and as fast as possible.
Reports published by the European Environment Agency of the European Union and the \ac{ipcc} of the United Nations indicate that transportation is one of the categories with substantial CO2 emissions~\cite{lee2023climate2}.
Cars cause over 70\% of the green house gas emissions, indicating that there is a big potential for savings.
More and more societies worldwide are aware of this problem and search for solutions to adapt to the new changes that comes with the climate change and try to decrease the impact of the global climate crisis.
This is why cities worldwide try to increase the modal share of bicycle traffic and by doing so, decrease the usage of motorized private transport.
The heavy usage of motorized private transport in urban areas also cause other problems.
Firstly, there are the motor vehicle traffic crashes with an annual death toll of over 1.3 million and an annual injury toll of over 78.2 million.~\cite{bhalla2014transport}.
Secondly, there are health issues that are indirectly caused by motorized private transport.
Transportation is one of the main causes for air pollution in urban areas in Europe~\cite{european2019european} and it is estimated, that around 8 million people die due to health issues that are connected to air pollution every year~\cite{forouzanfar2016global}. 
Motorized private transport causes air pollution in various ways.
Cars with an internal combustion engine produce (green house) gases such as CO2 or NOx as a byproduct from burning fossil fuels and particulate matter from the friction of the brakes and other car parts, as well as the friction between the tires and the street.
Thirdly, noise pollution, to which car traffic undoubtedly also contributes significantly, also causes serious health issues~\cite{khreis2016health}.
Sleep and stress disorders, adverse reproductive outcomes are just some of the problems related to traffic noise.


\section{Problem Statement}
\label{sec:problem}
This, however, puts cyclists, city planners and politicians into a difficult spot.
Since cities were developed with the motorized private transport as the main transportation mode in mind, the traffic infrastructure highly favors the usage of cars.
Because of that, cyclists often have a hard time commuting, since they have to cycle on roads that were made for cars, which often puts them into dangerous and stressful situations.
Although crash statistics do not convey an increased danger for cyclists (CITATION NEEDED), studies have shown, that the perceived safety of commuting with a bicycle is very low (CITATION NEEDED) and one of the main reasons why people prefer other transportation modes to cycling (CITATION NEEDED).
It is up to city planners to improve the traffic infrastructure for cyclists, but it is not easy to pinpoint the most problematic areas, that need to be prioritized for them to be fixed, which is needed because of the limited resources available to the city planning department.
Besides, city planners are bound by city planning codes and rules, which oftentimes also reflect the car-centric approach.
This passes the responsibility to the politicians, who need to change the city planning codes and rules and also increase the funds for bicycle infrastructure.
For this, however, they need more than anecdotal evidence, that the perceived safety in bicycle traffic is low, so that they can change the legislation and increase funding in favor of bicycle infrastructure.

To enable city planners and politicians to do evidence-based decision-making, they need a) to be aware of the problems in bicycle traffic and b) to know where and how to improve the bicycle infrastructure.
There are surveys that show that the perceived safety in bicycle traffic is low (MULTIPLE CITATIONS)
DO I NEED TO GIVE EXAMPLES HERE FOR SURVEYS?

\section{Contributions of this Thesis}
\label{sec:contributions}

\section{Structure of this Thesis}
\label{sec:structure}