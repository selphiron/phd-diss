\cleardoublepage
\chapter{Introduction}
\label{cha:introduction}
The global climate crisis is and will be one of mankind's most vital challenges in the first half of the 21\textsuperscript{st} century and the effects thereof are already observable.
The frequency of extreme weather conditions and natural disasters such as long periods of precipitation leading to floods, a lack of precipitation leading to droughts and storms is rising.
Such disasters have short and long-term effects on the agricultural production of the affected areas leading to famine, conflicts, and mass migration~\cite{kaczan2020impact}.
To prevent further aggravation of the effects, and stop or at least slow down global warming, it is important to understand the root cause, so possible solutions can be derived and implemented. 
According to the \ac{ipcc}, the surface temperature of the globe has risen by 1.1°C during the period 2011-2020 when compared to the period of 1850-1900~\cite{lee2023climate}.
There is a consensus that this increase in the surface temperature is caused by anthropogenic emissions stemming from human activities such as deforestation and the burning of fossil fuels thus releasing $CO_{2}$ and $NO_{x}$ to the atmosphere~\cite{archer2010climate}, both of which are greenhouse gases~\cite{grewe2019contribution}.
If the current trend of greenhouse gas emissions continues, the Paris Agreement's~\cite{un2015paris} goal to keep the increase of the surface temperature of the globe under 2°C will not be met~\cite{noah2023data}.
That means that to stop or at least slow down global warming, a decrease in greenhouse gas emissions is needed as much and as fast as possible.

Reports published by the European Environment Agency of the European Union and the \ac{c} indicate that transportation is one of the categories with substantial CO2 emissions~\cite{lee2023climate2}:
Cars cause over 70\% of greenhouse gas emissions in the transport sector, which accounts for 23\% of global energy-related $CO_{2}$ emissions~\cite{jaramillo2022transport}, indicating that there is a big potential for savings.
More and more societies worldwide are aware of this problem and are searching for solutions to adapt to the new changes that come with climate change and try to decrease the impact of the global climate crisis.
This is why cities worldwide try to increase the modal share of bicycle traffic and by doing so, decrease the usage of motorized private transport.
The heavy usage of motorized private transport in urban areas also causes other problems.

First, there are motor vehicle traffic crashes with an annual death toll of over 1.3 million and an annual injury toll of over 78.2 million worldwide~\cite{bhalla2014transport}.
Additionally, these crashes can cause long-lasting psychological traumas for everyone involved, such as witnesses or police officers, health workers, and firefighters who have to document the crash, help injured people, and rescue people who are stuck in the car wreck.

Second, there are health issues that are indirectly caused by motorized private transport.
Transportation is one of the main causes of air pollution in urban areas in Europe~\cite{european2019european} and it is estimated, that around 8 million people die due to health issues that are connected to air pollution every year~\cite{forouzanfar2016global}. 
Motorized private transport causes air pollution in various ways.
Cars with an internal combustion engine produce (greenhouse) gases such as CO2 or NOx as a byproduct from burning fossil fuels and particulate matter from the friction of the brakes and other car parts, as well as the friction between the tires and the street.

Third, noise pollution, to which car traffic also contributes significantly, also causes serious health issues~\cite{khreis2016health}.
Sleep and stress disorders and adverse reproductive outcomes are just some of the problems related to traffic noise.

Other problems of motorized private transport in urban areas and car-centric city planning are urban sprawl, which leads to less walkable cities~\cite{patacchini2009urban} and thus a more sedentary lifestyle, which in turn causes health issues stemming from a lack of physical activities, soil sealing, which damages biodiversity and the ecosystem~\cite{tobias2018soil} and many more ~\cite{bozovic2021non,pritchard2022maas,mayers2020whose}.  


\section{Problem Statement}
\label{sec:problem}
The shift in urban development priorities from public transportation to private motor vehicles in the 1970s has resulted in a transportation infrastructure that is heavily oriented towards automobile usage.
Because of that, cyclists have to cycle on roads that were made for cars, which often puts them into dangerous and stressful situations.
Although crash statistics do not convey an increased danger for cyclists~\cite{juhra2012bicycle}, studies have shown, that the perceived safety of cycling is very low and one of the main reasons why people prefer other transportation modes to cycling~\cite{horton2016fear}.
This poses a challenge for (potential) cyclists, city planners, and politicians alike;
Individuals who wish to engage in cycling must surmount their perceived lack of safety and comfort when commuting by bicycle, thereby avoiding the necessity of relying on motor vehicles. 
It is up to city planners to improve the traffic infrastructure for cyclists, but it is not easy to pinpoint the most problematic areas, that need to be prioritized for them to be fixed, which is needed because of the limited resources available to the city planning department.
Besides, city planners are bound by city planning codes and rules, which oftentimes also reflect the car-centric approach and hinder them from changing the traffic infrastructure in a way that would benefit cyclists.
This passes the responsibility to the politicians, who need to change the city planning codes and rules and also increase funding for bicycle infrastructure.
For this, however, they need more than anecdotal evidence, that the perceived safety in bicycle traffic is low so that they can change the legislation and increase funding in favor of bicycle infrastructure.

To enable city planners and politicians to do evidence-based decision-making, they need a) to be aware of the problems in bicycle traffic, b) that a lot of people are interested in improved bicycle traffic safety and c) to know where and how to improve the bicycle infrastructure.
Cyclists need good bicycle infrastructure that increases both the safety and comfort of cycling so that more and more people choose the bicycle over other modes of transport.
If the existing bicycle infrastructure is not good enough, they need to know which street sections are dangerous so that they can avoid them or be extra careful, when they have to pass through them.
Additionally, it would motivate cyclists to know that the problem is worked on and to keep the demand for a good bicycle infrastructure high. 

\section{Contributions of this Thesis}
\label{sec:contributions}
In this thesis, we address the problems mentioned above by fulfilling three needs:
First, we conceptualize and implement a cycling trip crowdsourcing platform, that also collects information about near miss incidents.
Second, we derive the surface quality of bicycle roads.
Third, we improve the bicycle simulation of a widely used traffic simulation software.

\subsection{SimRa: A Platform for Crowdsourcing Cycling Trip Data and Detecting Near Miss Incidents in Bicycle Traffic}
\label{subsec:simra_contribution}
The SimRa\footnote{SimRa is a German acronym for ``Sicherheit im Radverkehr´´, which translates to Safety in Bicycle Traffic.} platform comprises all things related to the collection, storage, and analysis of crowdsourced cycling data.
The data acquisition relies on a free app, which is available on both major smartphone operating systems Android and iOS, installed on the smartphones of participating cyclists.
This app collects data and detects near miss incidents\footnote{In the remainder of this thesis, we will also refer to them as ‘‘incidents’’.} during bicycle trips, lets users add comments or labels, and anonymizes the data before uploading it to the SimRa servers.
Initially, the detection of incidents used a heuristic, where the accelerometer data of the recorded bicycle trip were analyzed and the biggest differences in consecutive values were considered an incident.
We then built a second incident detection option that relies on a deep learning model.
The anonymized data comprises information on cyclist routes, incidents and user demographics, as well as some aggregated bicycle trip statistics.
Finally, the collected data gets processed and analyzed continuously to gain insights into dangerous street segments and intersections.
For this, we have developed one approach for interactive exploratory data analysis based on a web application and one for confirmatory data analysis which automatically derives a ‘‘dangerousness’’ score per street segment and intersection. 

We have published this contribution in~\cite{karakaya2020simra,karakaya2022cyclesense}.

\subsection{Deriving Road Surface Quality from Cycling Trip Data}
\label{subsec:road_surface_contribution}
Since the SimRa dataset also contains implicit information about the surface quality of the roads the bicycle trip was recorded on as part of the accelerometer sensor track.
While the automatic detection of incidents views the vibrations stemming from the uneven surface of the roads as noise data, here we can use them to derive the ‘‘bumpiness’’ of the road.
One major challenge, though, is the heterogeneity  of the user base who contribute to the SimRa dataset.
Different bicycle types and the place of the smartphone during the bicycle trip have a big impact on how intensive the vibrations caused by the road surface are detected by the accelerometer sensors.
The bicycle trip data contain information about the bicycle type and the position of the smartphone during the bicycle trip, however, this information can be given voluntarily by the user.
Heterogeneous cyclists also mean different styles of cycling.
While some may cycle rather quietly and relaxed, others may tend to cycle fast and sporty, which can also influence the resulting motion sensor readings.
Additionally, a wide variety of different smartphone models result in many different hardware and software, when it comes to reading motion sensor data.
The aforementioned aspects can result in different ‘‘bumpiness’’ values, from many bicycle trips that go through the same road.
This is why our approach considers the relative road surface quality that makes it possible to successfully compare different streets with each other.

We have published this contribution in~\cite{karakaya2023crowdsensing}.

\subsection{Cyclist Model for SUMO}
\label{subsec:sumo_contribution}
The SimRa dataset contains the GPS trace of the bicycle trip as well as the timestamp for each tracked GPS point.
The main reason to include that data is to be able to show the recorded trip to the user and to place incidents in the correct position.
Additionally, by knowing where the trip has been through, we can identify dangerous and safe areas in bicycle traffic.
However, we can also extract the longitudinal movement profile, that is the acceleration, deceleration, and maximum velocity of cyclists, as well as their behavior at intersections.
With this, we improve the bicycle simulation model of SUMO~\cite{lopez2018microscopic}, which simulates bicycles as slower cars or faster pedestrians by default, which is obviously far from being realistic.
To do so, we first separate the bicycle trips into slow, medium, and fast trips to distinguish between slow, medium, and fast cyclists.
A comparison of the longitudinal movement and the left-turn behavior at intersections shows how unrealistic the default bicycle simulation of SUMO is.
Based on the three groups of bicycle trips, we each create a model for slow, medium, and fast cyclists in SUMO and show the improvement compared to the default bicycle model of SUMO.

We have published this contribution in~\cite{karakaya2022realistic,karakaya2023achieving}.

\section{Structure of this Thesis}
\label{sec:structure}

This thesis consists of three parts.
Part I: Foundations begins with this chapter and proceeds with background information and related work on the topics of cycling comfort, cycling safety, crowdsourcing, crowdsensing, and citizen science (\Cref{cha:background}).
Part II: Improving Safety in Bicycle Traffic contains our three main contributions.
We first start with the SimRa platform, which is a crowdsourcing-based citizen science project, that gathers records of cycling trips and incident data.
There, we also describe CycleSense, our deep learning model for automatically detecting incidents from cycling trip data (\Cref{cha:cyclesense}).
Second, we describe our approach for deriving the road surface quality from cycling trip data and how to integrate this approach into a cycling app.
Additionally, we show how a visualization of the results can look like (\Cref{cha:cyclequality}).
Third, we present how we improve the existing cyclist model of the urban traffic simulation software \ac{sumo} (\Cref{cha:sumo}).
Finally, Part III: Conclusions concludes this thesis with a summary of our contributions (\Cref{cha:summary}) and gives a discussion and outlook of our work (\Cref{cha:discussion_and_outlook}).
