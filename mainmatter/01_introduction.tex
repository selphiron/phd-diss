\cleardoublepage
\chapter{Introduction}
\label{cha:introduction}
The global climate crisis is and will be one of mankind's main challenges in the coming decades.
More and more societies worldwide are aware of this problem and search for solutions to adapt to the new changes that comes with the climate change and try to decrease the impact of the global climate crisis.
For the latter, the main approach is to reduce greenhouse gas emissions such as $CO_{2}$ and $NO_{x}$.
This is why cities worldwide try to increase the modal share of bicycle traffic and by doing so, decrease the usage of individual motorized transport, which is one of the main greenhouse gas emitters in urban areas (CITATION NEEDED).
This, however, puts cyclists, city planners and politicians into a difficult spot.
Since cities were developed with the individual motorized transport as the main transportation mode in mind, the traffic infrastructure highly favors the usage of cars.
Because of that, cyclists often have a hard time commuting, since they have to cycle on roads that were made for cars, which often puts them into dangerous and stressful situations.
Although crash statistics do not convey an increased danger for cyclists (CITATION NEEDED), studies have shown, that the perceived safety of commuting with a bicycle is very low (CITATION NEEDED) and one of the main reasons why people prefer other transportation modes to cycling (CITATION NEEDED).
It is up to city planners to improve the traffic infrastructure for cyclists, but it is not easy to pinpoint the most problematic areas, that need to be prioritized for them to be fixed, which is needed because of the limited resources available to the city planning department.
Besides, city planners are bound by city planning codes and rules, which oftentimes also reflect the car-centric approach.
This passes the responsibility to the politicians, who need to change the city planning codes and rules and also increase the funds for bicycle infrastructure.
For this, however, they need more than anecdotal evidence, that the perceived safety in bicycle traffic is low, so that they can change the legislation and increase funding in favor of bicycle infrastructure.

\section{Problem Statement}
\label{sec:problem}
To enable city planners and politicians to do evidence-based decision-making, they need a) to be aware of the problems in bicycle traffic and b) to know where and how to improve the bicycle infrastructure.
There are surveys that show that the perceived safety in bicycle traffic is low (MULTIPLE CITATIONS)
DO I NEED TO GIVE EXAMPLES HERE FOR SURVEYS?

\section{Contributions of this Thesis}
\label{sec:contributions}

\section{Structure of this Thesis}
\label{sec:structure}